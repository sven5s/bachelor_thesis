One of the big questions of mankind is, why we are here. And the correspondending 
question in physics is, why more matter than anti-matter exists in the universe. 
This thesis cannot clear the question but it can be a part of the big picture.\\
The \texttt{LHC} (Large Hadron Collider) is one of the biggest experiments of 
humanity. This accelerator collides proton with protons, lead nuclei with 
protons or lead nuclei with lead nuclei. One detector of this big apparatus 
is the \texttt{LHCb}. The \texttt{LHCb} is a try to find the answer of the 
above question. It examines asymmetries in the decay of matter and 
anti-matter mainly for proton-proton-collissions and so deviations 
from the Standard Model. The Standard Model is a theory that describes the 
fundamental interactions between and the elementary particles itself very 
well. These deviations can result in additions to the Standard Model to consider 
a asymmetry between matter and anti-matter, and so lead to an answer for the 
matter surplus.\\
For this the results from the experiment have to be compared to the 
theoratical results. But simulations are needed for the theoretical results because 
the theories are very complicated.
This comes from the different interactions that happens through a particle 
collision. A lot of particles are created and then they decay. On of the simulation 
tools is \texttt{Sherpa}{\cite{sherpa}}. This program uses Monte-Carlo techniques 
to obtain results.\\
One possible baryon that can be created is the \(\Lambda_c^+\). It is the 
lightest baryon that contains a charm quark. The charm quark is like the 
up quark, that builds neutrons and protons together with the down quark, but 
heavier.\\
For a good comparison between the experimental results and the theoretical 
simulations it is important to implement brayons and not only bare quarks in the 
simulation. The behavior of quarks in bound states is different to free quarks 
that didn't exists in nature. So the hadronization and the the decay of the 
hadrons are important processes. This processes have to be recognized to 
get good theoratical predictions.
\par
This thesis has the goal to improve the simulation through a better implementation 
of the semileptonic \(\Lambda_c^+\) decays.

