One of the big questions of mankind is, why are we here. And the correspondending 
question is in physics is, why exists more matter than anti-matter in the universe. 
This thesis can't clear the question but it can be a part of the big picture.\\
The \texttt{LHC} (Large Hadron Collider) is one of the biggest experiments of 
humanity. This accelerator collides proton with protons, lead nuclei with 
protons or lead nuclei with lead nuclei. One detector of this big apparatus 
is the \texttt{LHCb}. The \texttt{LHCb} is a try to find the answer of the 
above question. It examines assymetries in the decay of matter and 
anti-matter mainly for lead-lead-collissions. And so deviations 
from the Standard Model. The Standard Model is a theory that describes the 
fundamental interactions between and the elementary particles itself very 
well. These deviations can result in additions to the Standard Model to consider 
a asymmetry between matter and anti-matter. And so lead to an answer for the 
matter surplus.\\
For this the results from the experiment has to be compared to the 
theoratical results. But simulations are needed for the theoratical results. 
This comes from the different interactions that happens through a particle 
collision. A lot of particles are created and decay. On of the simulation 
tools is \texttt{Sherpa}. This program use Monte-Carlo techniques to obtain 
results.\\
One possible baryon that can be created is the \(\Lambda_c^+\). It is the 
lightest baryon that contains a charm quark. The charm quark is like the 
up quark that builds together with the down quark neutrons and protons but 
heavier.\\
For a good comparison between the experimental results and the theoratical 
simulations is it important to implement as much particles in the simulation 
as possible. This leads in a better description of the reaction and so in better 
theoretical results.
\par
This thesis has the goal to improve the simulation through a better implementation 
of the \(\Lambda_c^+\) decays.
