\section{Decays Data}
\texttt{Sherpa} use for the decays from all kinds of particles the decay channels 
and branching ratios from the \texttt{Decaysdata.db}. This database has to be 
updated manually with data from the Particle Data Group (PDG) because there 
exists no automation for this work. Also data from other sources are included, 
e.g. EvtGen.\\
Good results need actual data. The first part is to update the branching ratios 
and deyays. For the \(\Xi(1690)\) was an implementation not possible because 
there exists to few events about futher decays. The conclusion of different events 
needs a lot of attention. For some events like \(\Lambda_c^+ \rightarrow  
\Sigma(1385)^- + \pi^+ + \pi+\) only the channel \(\Sigma(1385) \rightarrow 
\Lambda + \pi\) was recognized. So a division with the BR(\(\Sigma(1385) 
\rightarrow \Lambda + \pi\)) was needed because \texttt{Sherpa} handle further 
decays and consider all different decays of the \(\Sigma(1385)^-\). Table 
{\eqref{ta:changes-full}} was revealed with this knowledge. An abstract is 
visible in {\eqref{ta:changes-decays}}.
\begin{longtable}{| c | c | c |}
  \caption{Extract of the changes in the Decays.dat from the \(\Lambda_c^+\)}\label{ta:changes-decays}\\ 
  \hline
  \input{Decays.dat.changes}
\end{longtable}
In the case of \(\Lambda_c^+ \rightarrow P^+ \pi^+ + \pi^-\) was the decay 
already included in \(\Lambda_c^+ \rightarrow P^+ + f(0980)\). \(K_b\) was 
removed from the PDG and only \(K_s\) exists. For \(\Lambda_c^+ \rightarrow 
\Lambda + \eta + \pi^+\) give the difference between \(\Lambda + \pi^- + 
\pi + \pi^+ + \pi^+\) and  \(\Sigma(1385)^+ + \eta\) the right value because 
an \(\eta\) decays in \(\pi^+ + \pi^-\) and a \(\Sigma(1385)\) to 
\(\Lambda + \pi\). But the decay of the \(\Sigma(1385)\) is a separate channel.\\
These decays are selected because the full list would be too long and so 
it is visible that some channel becomes likelier. But as an compensation other 
process has to become less probable. These process are in most cases very 
similiar and redistribution rests upon a better identification of the final 
decay states and a better particle reconstruction. One of the important changes 
is the increase of the branching ratios from the semileptonic decays by nearly 
\(1 \%\).\\
Other important process that didn't changed are shown in table{\eqref{ta:ndecays}}
\begin{longtable}{| c | c | c | c |}
  \caption{Decays in a neutron in the Decays.dat from the \(\Lambda_c^+\)}\label{ta:ndecays}\\ 
  \hline
  \input{Decays.dat.neutron}
\end{longtable}
The branching for these semileptonic decays only simulated. The reason is that 
most of the modern detectors can't detect neutrons very well. This comes from 
the neutral electric charge and the long lifetime of the neutron. An improved 
measurement would be recommended because this processes are important for 
the form factor calculation. The neutron is often considered as the final 
decay state oif the \(\Lambda_c^+\).\\
All these changes leads to a sum of all branching ratios from \(87,98\%\). 
This is a pretty good value and very close to \(100\%\). This means that 
the decay is very good abstracted.

\section{Form Factor conversion}
The two formulas in {\eqref{eq:v-a}} show one popular parametrization for 
the V-A-Current another possible and popular writing is given in 
{\eqref{eq:v-a-q}}.
\begin{align}
  \langle B(p', s') | V_\mu | \Lambda_c(p, s) \rangle & = \bar{u}(p', s') 
  \left( f^V_1(q^2) \gamma_\mu + f^V_2(q^2)i\sigma_{\mu\nu}\frac{q^\nu}{m_{\Lambda_c}} + 
  f^V_3(q^2)\frac{q_\mu}{m_{\Lambda_c}} \right) u(p, s) \nonumber \\
  \langle B(p', s') | A_\mu | \Lambda_c(p, s) \rangle & = \bar{u}(p', s') 
  \left( f^A_1(q^2) \gamma_\mu + f^A_2(q^2)\i\sigma_{\mu\nu}frac{q^\nu}{m_{\Lambda_c}} + 
  f^A_3(q^2)\frac{q_\mu}{m_{\Lambda_c}} \right) \gamma^5 u(p, s) \label{eq:v-a-q}
\end{align}

In this notation is \(q = p - p'\). A convsersion formula is now needed for the 
different parametrization of the current. The equations in 
{\cite[Eq. 15]{form_factor_conversion}} give one direction for transformation. 
But most of the following form factors are in the form with p and p' and not 
with q. So the inversion of the given transformation would be the easiest 
way to get a decent formula. Another fact is that the current for an baryon 
baryon transition in \texttt{Sherpa} is already in the form like {\eqref{eq:v-a}}. 
The use of this parametrization let the implementation become a lot easier. 
The first step is to create a transformation matrix like in {\eqref{eq:initmat}}
\begin{equation}
  \Spvek{f^V_1; f^V_2; f^V_3; f^A_1; f^A_2; f^A_3} =
  \begin{bmatrix}
    1 & \frac{m_{\Lambda_c} + m_B }{2 m_{\Lambda_c}} & \frac{m_{\Lambda_c} + m_B }{2 m_B} & 0 & 0 & 0 \\
    0 & -\frac{1}{2} & -\frac{m_{\Lambda_c}}{2 m_B} & 0 & 0 & 0 \\
    0 & \frac{1}{2} & -\frac{m_{\Lambda_c}}{2 m_B} & 0 & 0 & 0 \\
    0 & 0 & 0 & 1 & -\frac{m_{\Lambda_c} - m_B }{2 m_{\Lambda_c}} & - \frac{m_{\Lambda_c} - m_B }{2 m_B} \\
    0 & 0 & 0 & 0 & -\frac{1}{2} & -\frac{m_{\Lambda_c}}{2 m_B} \\
    0 & 0 & 0 & 0 & \frac{1}{2} & -\frac{m_{\Lambda_c}}{2 m_B}
  \end{bmatrix}
  \cdot \Spvek{F_1; F_2; F_3; G_1; G_2; G_3} \label{eq:initmat}
\end{equation}
The block structure of the matrix is a mathemtaical manifestation of the 
independence of the vector and axial-vector part. This matrix can be splitted 
in two equations{\eqref{eq:splitmat}}.
\begin{align}
  \Spvek{f^V_1; f^V_2; f^V_3} & =
  \begin{bmatrix}
    1 & \frac{m_{\Lambda_c} + m_B }{2 m_{\Lambda_c}} & \frac{m_{\Lambda_c} + m_B }{2 m_B} \\
    0 & -\frac{1}{2} & -\frac{m_{\Lambda_c}}{2 m_B} \\
    0 & \frac{1}{2} & -\frac{m_{\Lambda_c}}{2 m_B} \\
  \end{bmatrix}
  \cdot \Spvek{F_1; F_2; F_3} \nonumber \\
  \Spvek{f^A_1; f^A_2; f^A_3} & =
  \begin{bmatrix}
    1 & -\frac{m_{\Lambda_c} - m_B }{2 m_{\Lambda_c}} & - \frac{m_{\Lambda_c} - m_B }{2 m_B} \\
    0 & -\frac{1}{2} & -\frac{m_{\Lambda_c}}{2 m_B} \\
    0 & \frac{1}{2} & -\frac{m_{\Lambda_c}}{2 m_B}
  \end{bmatrix}
  \cdot \Spvek{G_1; G_2; G_3} \label{eq:splitmat}
\end{align}
The matrices can be inverted if the determinant is unequal to zero. The 
determinant of both matrices are the same{\eqref{eq:det}}. This comes from the 
block structure with the zeroes in the first column.
\begin{equation}
  det (\dots) = \frac{m_{\Lambda_c}}{2 m_B} \label{eq:det}
\end{equation}
If the mass of the \(\Lambda_c^+\) is nonzero than the matrices are invertible. And obviously 
is this true. The matrix for \(G_i\) is really in the fashion of the one for 
\(F_i\). So the inverting has only be done ones. The matrices in {\eqref{eq:finmat}}
are calculated.
\begin{align}
  \Spvek{F_1; F_2; F_3} & =
  \begin{bmatrix}
    1 & \frac{3 m_{\Lambda_c} + m_B }{4 m_{\Lambda_c}} & -\frac{m_{\Lambda_c} + m_B }{4 m_B} \\
    0 & -1 & 1 \\
    0 & -\frac{m_B}{2 m_{\Lambda_c}} & -\frac{m_B}{2 m_{\Lambda_c}} \\
  \end{bmatrix}
  \cdot \Spvek{f^V_1; f^V_2; f^V_3} \nonumber \\
  \Spvek{G_1; G_2; G_3} & =
  \begin{bmatrix}
    1 & -\frac{3 m_{\Lambda_c} - m_B }{4 m_{\Lambda_c}} & \frac{m_{\Lambda_c} + m_B }{4 m_B} \\
    0 & -1 & 1 \\
    0 & -\frac{m_B}{2 m_{\Lambda_c}} & -\frac{m_B}{2 m_{\Lambda_c}} \\
  \end{bmatrix}
  \cdot \Spvek{f^A_1; f^A_2; f^A_3}  \label{eq:finmat}
\end{align}
And finally for every form factor a transformation formula{\eqref{eq:fintrans}} 
can be obtained.
\begin{align}
  F_1 & = f^V_1 + \frac{m_{\Lambda_c} + m_B }{4} \left( \frac{3 f^V_2}{m_{\Lambda_c}} - \frac{f^V_3}{m_B} \right) \nonumber \\
  F_2 & = - f^V_2 + f^A_3 \nonumber \\
  F_3 & = - \frac{m_B}{2 m_{\Lambda_c}} \left(f^V_2 + f^V_3 \right) \nonumber \\
  G_1 & = f^A_1 + \frac{m_{\Lambda_c} - m_B }{4} \left( - \frac{3 f^V_2}{m_{\Lambda_c}} + \frac{f^V_3}{m_B} \right) \nonumber \\
  G_2 & = - f^A_2 + f^A_3 \nonumber \\
  G_3 & = - \frac{m_B}{2 m_{\Lambda_c}} \left(f^A_2 + f^V_3 \right) \label{eq:fintrans}
\end{align}
The basic class FormFactor\_Base from VA\_B\_B\_FFs in \texttt{HADRONS++/Current\_Library/VA\_B\_B.H} 
is now extend with additional inline functions for the transformation. To use 
every time the same parametrization and didn't mix them.

\section{Prevoius Form Factor implementation}
At first it is important to know wich current parametrization is already used 
by the \texttt{VA\_B\_B.C} in \texttt{Sherpa}. With this information the right 
form can be choosen.
For a natural transition between two baryons the different coefficient
{\eqref{eq:c-coeff}} are computed. A natural transition means that the decayer 
and the daughter baryon have the same parity.
\begin{align}
  c_{R1} & = V_1 - A_1 \nonumber \\
  c_{L1} & = -V_1 - A_1 \nonumber \\
  c_{R2} & = V_2 - A_2 \nonumber \\
  c_{L2} & = -V_2 - A_2 \nonumber \\
  c_{R3} & = V_3 - A_3 \nonumber \\
  c_{L3} & = -V_3 - A_3 \label{eq:c-coeff}
\end{align}
\(V_i\) and \(A_i\) comes from the choosen from factor model. The following 
current{\eqref{eq:b-curr}} is sumed ove all possible helicities from the 
decaying h\_0 and the daughter baryon h\_1.
\begin{align}
\begin{split}
  V_\mu &= L_\mu(1, h_1, 0, h_0, c_{R1}, c_{L1}) + \\
        & \frac{p_{0, \mu}}{m_0} \cdot Y(1, h_1, 0, h_0, c_{R2}, c_{L2}) + \\
        & \frac{p_{1, \mu}}{m_1} \cdot Y(1, h_1, 0, h_0, c_{R3}, c_{L3}) \label{eq:b-curr}
\end{split}
\end{align}
The definition of \(L_\mu\) is given in \cite[Eq. A.96]{diploma} and the Y in 
\cite[Eq. A.94]{diploma}. They are visible again in {\eqref{eq:LY}}.
\begin{align}
  L_\mu(1, h_1, 0, h_0, c_{R1}, c_{L1}) & = \bar{u}(p_1, h_1)\gamma_\mu \left( c_{R1} P_R + c_{L1} P_L \right) u(p_0, h_0) \nonumber \\
  Y(1, h_1, 0, h_0, c_{R2}, c_{L2}) & = \bar{u}(p_1, h_1) \left( c_{R2} P_R + c_{L2} P_L \right) u(p_0, h_0) \nonumber \\
  Y(1, h_1, 0, h_0, c_{R3}, c_{L3}) & = \bar{u}(p_1, h_1) \left( c_{R3} P_R + c_{L3} P_L \right) u(p_0, h_0) \label{eq:LY}
\end{align}
The projection operators that are used in {\eqref{eq:LY}} are defined in {\eqref{eq:proj}}
\begin{align}
  P_R = \frac{1}{2} \left( 1 + \gamma_5 \right) \nonumber \\
  P_L = \frac{1}{2} \left( 1 - \gamma_5 \right) \label{eq:proj}
\end{align}
This all equations together form a current wich is similiar to {\{\eqref{eq:v-a}}.
This can also easily been seen through the signs of the \(\gamma_5\) and the 
signs of the \(c_R\) and \(c_L\).
All other current parametrization have to be converted to observe the 
existing behavior.

\section{Observables}
Two observables are adapted from simulations of the \texttt{BELLE} experiment 
to compare the different form factors.
The first is \(q^2\) which is defined in {\eqref{eq:q2}}.
\begin{equation}
  q^2 = \left( p_{\Lambda_c} - p_B \right)^2 \label{eq:q2}
\end{equation}
And the second is the recoil of the W in {\eqref{eq:w}}
\begin{equation}
  w = \frac{p^2_{\Lambda_c} +  p^2_B - \left( p_{\Lambda_c} - p_B \right)^2}
  {2 \cdot \sqrt{p^2_{\Lambda_c}} \cdot \sqrt{p^2_B}} \label{eq:w}
\end{equation}

\section{Form Factor Models}
\subsection{Covariant Confined Quark Model (CCQM)}
The idea behind the covariant confined quark model is to use two loops 
Feynman diagrams with free quark propagators. The high energy behavior of 
the loop integrations is soften. This model was developed for mesons but 
is extended to baryons. It is also possible to successfully calculate tetraquark 
states with this theory.\\
For the transition to \(\Lambda + l^+ + \nu_l\) the paper{\cite{CCQM_L}} was 
considered and for \(n + l^+ + \nu_l\) the paper{\cite{CCQM_N}}.
The parametrization of the form factor is given by {\eqref{eq:ccqmff}}
\begin{equation}
  f(q^2) = \frac{F(0)}{1 - a s + b s^2} \text{ with } s = \frac{q^2}{m_{\Lambda_c}} \label{eq:ccqmff}
\end{equation}
The parameters F(0), a and b are taken from the numerical results of {\cite{CCQM_L}}.

\subsection{Non relativistic Quark Model (NRQM)}
Baryons with a heavy quark possesses a special symmetry. This symmetry is 
called heavy quark symmetry. This is based on works from Isgur and Wise. The name 
of the theory behind that is heavy quark effective theory (HQET). The main impact 
to the characteristics of the baryon results from the degrees of freedom of the 
light quarks and are independent from the degrees of freedom of the heavy quark.\\
This form factor can be used to calculate transistions to excited \(\Lambda\)
baryons {\cite{NRQM}}. The parametrization of the form factors {\eqref{eq:nrqmff}} a more complicated compared 
to the other ones.
\begin{align}
  F &= \left(a_0 + a_2 q^2 + a_4 q_4\right) e^{- \frac{3 m_\sigma^2 p_\Lambda^2}
  {2 m_\Lambda^2 \alpha_{\lambda \lambda'}^2} } \nonumber\\
  p_\Lambda & = \frac{1}{2 m_\Lambda}  \lambda^\frac{1}{2}(m_{\Lambda_c}^2, m_\Lambda^2, q^2) \nonumber \\
  \lambda(x, y, z) & = x^2 + y^2 + z^2 - 2xy - 2yz - 2zx \text{(triangle function)} \nonumber \\
  \alpha_{\lambda \lambda'} & = \sqrt{\frac{\alpha_\lambda^2 + \alpha_{\lambda'}^2 }{2}} \label{eq:nrqmff}
\end{align}
In equation{\eqref{eq:nrqmff}} is \(m_\sigma\) the mass of the light quark 
obtained from {\cite[p. 13/I]{NRQM}} and the \(\alpha_\lambda\) are size parameters of 
the baryons from {\cite[p. 13/II]{NRQM}}.

\subsection{Light-Cone Sume Rule (LCSR)}
The light-cone sume rule is a vary famous technique in the class of QCD sum 
rules. The basic idea is that the vacuum condensates carry no momentum. The 
light-cone expansion is used with increasing twist. With this model was the 
transition into \(\Lambda + l^+ + \nu_l\) in {\cite{LCSR}}  computed.\\
The parametrization{\eqref{eq:lcsrff}} is through the same values for the 
first two axial-vector and vector form factors very simple
\begin{align}
  f^V_1 & = f^A_1 \nonumber \\
  f^V_2 & = f^A_2 \nonumber \\
  f_i(q^2) & = \frac{f_i(0)}{a_2 s^2 + a_1 s + 1 } \text{ with } s = \frac{q^2}{m_{\Lambda_c}} \label{eq:lcsrff}
\end{align}
\(f_i(0)\), \(a_2\) and \(a_1\) are parameters of this parametrization. Only the 
first two factors are nonzero.

\subsection{Relativistic Quark Model (RQM)}
The relativistic quark model{\cite{RQM}} is based on the diquark wave function 
and the baryon wave function of the bound quark-diquark state. The calculation 
were done with relativistic quasipotential equation of the Schr\"{o}dinger type.
All computation were relativistically done.\\
It can predict senmileptonic transitions into n as well as into \(\Lambda\).
The parametrization{\eqref{eq:rqmff}} was done until a very high order of the 
\(q^2\).
\begin{equation}
  F(q^2) = \frac{F(0)}{1 - \sigma_1 s + \sigma_2 s^2 + \sigma31 s^3 + \sigma_4 s^4} 
  \text{ with } s = \frac{q^2}{m_{\Lambda_c}} \label{eq:rqmff}
\end{equation}

\subsection{QCD Sum Rule (QCDSR)}
In the article {\cite{QCDSR}} are nonperturbative aspects used. The approach of 
the QCD sum-rule is the expansion in local operators. Here is also the HQET usd 
to reduce the complexity. \(\Lambda + l^+ + \nu_l\) is considered in the article.\\
In this Model two different parametrizations exists. The pole parametrization 
{\eqref{eq:qcdsrff-pp}} is used like in the other models. But this pole 
parametrization is a lot simpler through relations between the form factors.
\begin{align}
  f^A_1 & = - f^V_1 \nonumber \\
  f^A_2 & = f^V_2 \nonumber \\
  f^V_i(q^2) & = \frac{a_0}{a_1 - q^2} \label{eq:qcdsrff-pp}
\end{align} 
The other form of the parametrization  {\eqref{eq:qcdsrff}} is more complicated 
but through the exponential ansatz interesting.
\begin{align}
  f^A_1 & = - f^V_1 \nonumber \\
  f^A_2 & = - f^V_2 \nonumber \\
  f^V_1(q^2) & = e^{\frac{q^2 - a_1}{a_0}} \nonumber \\
  f^V_2(q^2) & = \frac{a_0}{a_1 - q^2} \label{eq:qcdsrff}
\end{align} 
The third form factor is in both cases zero.

