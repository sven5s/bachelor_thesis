\section{Decaysdata.db}
\texttt{Sherpa} use for the decays from all kinds of particles the decay channels 
and branching ratios from the \texttt{Decaysdata.db}. This database has to be 
updated manually with data from the Particle Data Group (PDG) because there 
exists no automation for this work. Also data from other sources are included, 
e.g. EvtGen.\\
Good results need actual data. The first part is to update the branching ratios 
and deyays. For the \(\Xi(1690)\) was an implementation not possible because 
there exists to few events about futher decays. The conclusion of different events 
needs a lot of attention. For some events like \(\Lambda_c^+ \rightarrow  
\Sigma(1385)^- + \pi^+ + \pi+\) only the channel \(\Sigma(1385) \rightarrow 
\Lambda + \pi\) was recognized. So a division with the BR(\(\Sigma(1385) 
\rightarrow \Lambda + \pi\)) was needed because \texttt{Sherpa} handle further 
decays and consider all different decays of the \(\Sigma(1385)^-\). Table 
{\eqref{ta:changes-full}} was revealed with this knowledge. An abstract is 
visible in {\eqref{ta:changes-decays}}.
\begin{longtable}{| c | c | c | c |}
  \caption{Extract of the changes in the Decays.dat from the \(\Lambda_c^+\)}\label{ta:changes-decays}\\ 
  \hline
  \input{Decays.dat.changes}
\end{longtable}
In the case of \(\Lambda_c^+ \rightarrow P^+ \pi^+ + \pi^-\) was the decay 
already included in \(\Lambda_c^+ \rightarrow P^+ + f(0980)\). \(K_b\) was 
removed from the PDG and only \(K_s\) exists. For \(\Lambda_c^+ \rightarrow 
\Lambda + \eta + \pi^+\) give the difference between \(\Lambda + \pi^- + 
\pi + \pi^+ + \pi^+\) and  \(\Sigma(1385)^+ + \eta\) the right value because 
an \(\eta\) decays in \(\pi^+ + \pi^-\) and a \(\Sigma(1385)\) to 
\(\Lambda + \pi\). But the decay of the \(\Sigma(1385)\) is a separate channel.\\
These decays are selected because the full list would be too long and so 
it is visible that some channel becomes likelier. But as an compensation other 
process has to become less probable. These process are in most cases very 
similiar and redistribution rests upon a better identification of the final 
decay states and a better particle reconstruction. One of the important changes 
is the increase of the branching ratios from the semileptonic decays by nearly 
\(1 \%\).\\
Other important process that didn't changed are shown in table{\eqref{ta:ndecays}}
\begin{longtable}{| c | c | c | c |}
  \caption{Decays in a neutron in the Decays.dat from the \(\Lambda_c^+\)}\label{ta:ndecays}\\ 
  \hline
  \input{Decays.dat.neutron}
\end{longtable}
The branching for these semileptonic decays only simulated. The reason is that 
most of the modern detectors can't detect neutrons very well. This comes from 
the neutral electric charge and the long lifetime of the neutron. An improved 
measurement would be recommended because this processes are important for 
the form factor calculation. The neutron is often considered as the final 
decay state o the \(\Lambda_c^+\).

\section{Form factor conversion}
The two formulas in {\eqref{eq:v-a}} show one popular parametrization for 
the V-A-Current another possible and popular writing is given in 
{\eqref{eq:v-a-q}}.
\begin{align}
  \langle B(p', s') | V_\mu | \Lambda_c(p, s) \rangle & = \bar{u}(p', s') 
  \left( f^V_1(q^2) \gamma_\mu + f^V_2(q^2)i\sigma_{\mu\nu}\frac{q^\nu}{m_{\Lambda_c}} + 
  f^V_3(q^2)\frac{q_\mu}{m_B} \right) u(p, s) \nonumber \\
  \langle B(p', s') | A_\mu | \Lambda_c(p, s) \rangle & = \bar{u}(p', s') 
  \left( f^A_1(q^2) \gamma_\mu + f^A_2(q^2)\i\sigma_{\mu\nu}frac{q^\nu}{m_{\Lambda_c}} + 
  f^A_3(q^2)\frac{q_\mu}{m_B} \right) \gamma^5 u(p, s) \label{eq:v-a-q}
\end{align}

In this notation is \(q = p - p'\). A convsersion formula is now needed for the 
different parametrization of the current. The equations in 
{\cite[Eq. 15]{form_factor_conversion}} give one direction for transformation. 
But most of the following form factors are in the form with p and p' and not 
with q. So the inversion of the given transformation would be the easiest 
way to get a decent formula. Another fact is that the current for an baryon 
baryon transition in \texttt{Sherpa} is already in the form like {\eqref{eq:v-a}}. 
The use of this parametrization let the implementation become a lot easier. 
The first step is to create a transformation matrix like in {\eqref{eq:initmat}}
\begin{equation}
  \Spvek{f^V_1; f^V_2; f^V_3; f^A_1; f^A_2; f^A_3} =
  \begin{bmatrix}
    1 & \frac{m_{\Lambda_c} + m_B }{2 m_{\Lambda_c}} & \frac{m_{\Lambda_c} + m_B }{2 m_B} & 0 & 0 & 0 \\
    0 & -\frac{1}{2} & -\frac{m_{\Lambda_c}}{2 m_B} & 0 & 0 & 0 \\
    0 & \frac{1}{2} & \frac{m_{\Lambda_c}}{2 m_B} & 0 & 0 & 0 \\
    0 & 0 & 0 & 1 & -\frac{m_{\Lambda_c} - m_B }{2 m_{\Lambda_c}} & - \frac{m_{\Lambda_c} - m_B }{2 m_B} \\
    0 & 0 & 0 & 0 & -\frac{1}{2} & -\frac{m_{\Lambda_c}}{2 m_B} \\
    0 & 0 & 0 & 0 & \frac{1}{2} & -\frac{m_{\Lambda_c}}{2 m_B}
  \end{bmatrix}
  \cdot \Spvek{F_1; F_2; F_3; G_1; G_2; G_3} \label{eq:initmat}
\end{equation}
