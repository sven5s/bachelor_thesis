\section{Decaysdata.db}
\texttt{Sherpa} use for the decays from all kinds of particles the decay channels 
and branching ratios from the \texttt{Decaysdata.db}. This database has to be 
updated manually with data from the Particle Data Group (PDG) because there 
exists no automation for this work. Also data from other sources are included, 
e.g. EvtGen.\\
Good results need actual data. The first part is to update the branching ratios 
and deyays. For the \(\Xi(1690)\) was an implementation not possible because 
there exists to few events about futher decays. The conclusion of different events 
needs a lot of attention. For some events like \(\Lambda_c^+ \rightarrow  
\Sigma(1385)^- + \pi^+ + \pi+\) only the channel \(\Sigma(1385) \rightarrow 
\Lambda + \pi\) was recognized. So a division with the BR(\(\Sigma(1385) 
\rightarrow \Lambda + \pi\)) was needed because \texttt{Sherpa} handle further 
decays and consider all different decays of the \(\Sigma(1385)^-\). Table 
{\eqref{ta:changes-decays}} was revealed with this knowledge.
\begin{longtable}{| c | c | c | c |}
  \caption{Changes in the Decays.dat from the \(\Lambda_c^+\)}\label{ta:changes-decays}\\ 
  \hline
  \input{Decays.dat.changes}
\end{longtable}
In the case of \(\Lambda_c^+ \rightarrow P^+ pi^+ + pi^-\) was the decay 
already included in \(Lambda_c^+ \rightarrow P^+ + f(0980)\). \(K_b\) was 
removed from the PDG and only \(K_s\) exists. For \(Lambda_c^+ \rightarrow 
\Lambda + \eta + pi^+\) give the difference between \(\Lambda + \pi^- + 
\pi + \pi^+ + \pi^+\) and  \(\Sigma(1385)^+ + \eta\) the right value because 
an \(\eta\) decays in \(\pi^+ + \pi^-\) and a \(\Sigma(1385)\) to 
\(\Lambda + \pi\). But the decay of the \(\Sigma(1385)\) is a separate channel.

