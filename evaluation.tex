\section{Decays data}
With the old and the new decays data was made a simulation of the decay of the 
\(\Lambda_c^+\). For a good statistic 100000000 events was used. This number 
is a good comparison between runtime and number of events per bin.
\begin{figure}[h]
  \centering
  \includegraphics[width=0.45\textwidth]{../results/identified_states/comp_data/rivet-plots/MeasureLeptonsAndMoreDirectChildren/AntiElectron.pdf}
  \includegraphics[width=0.45\textwidth]{../results/identified_states/comp_data/rivet-plots/MeasureLeptonsAndMoreDirectChildren/AntiElectronEnergy.pdf}
  \caption{multiplicity and energy spectrum of the \(e^+\)} \label{gr:prim-ep}
\end{figure}
\begin{figure}[h]
  \centering
  \includegraphics[width=0.45\textwidth]{../results/identified_states/comp_data/rivet-plots/MeasureLeptonsAndMoreDirectChildren/AntiMyon.pdf}
  \includegraphics[width=0.45\textwidth]{../results/identified_states/comp_data/rivet-plots/MeasureLeptonsAndMoreDirectChildren/AntiMyonEnergy.pdf}
  \caption{multiplicity and energy spectrum of the \(\mu^+\)} \label{gr:prim-mup}
\end{figure}
In grpahics {\eqref{gr:prim-ep}} and {\eqref{gr:prim-mup}} you can see that 
the change of the branching ratio of the semileptonic decay in a \(\Lambda\) 
resulting in a nearly equal multiplicity but a significant different energy 
spectrum for both leptons. The sharp increase at the beginning of the muon 
energy spectrum comes from the bigger rest mass than the electron. This energy 
is needed to create a detectable on-shell muon. The two main process to create 
lepton are the semileptonic decay in \(\Lambda\) and n. Only the branching ratios 
for the \(\Lambda\) were changed. The higher rest mass of the \(\Lambda\) compared 
to the neutron lead to an significant energy shift. On the other hand the same 
number of leptons were created in this process. But compared to the other the 
branching ratio shifted a little bit to the muon, visible in the plots. 
The same result for the probability of the multiplicity of the muon proof that 
the changes in the branching ratios are proportional.\\
If the plots from the whole decays are considered it becomes clear that these 
edits also drastically changes the behavior in the final state.
\par 
\begin{figure}[h]
  \centering
  \includegraphics[width=0.45\textwidth]{../results/identified_states/comp_data/rivet-plots/MeasureLeptonsAndMoreDirectChildren/Pi+.pdf}
  \includegraphics[width=0.45\textwidth]{../results/identified_states/comp_data/rivet-plots/MeasureLeptonsAndMoreDirectChildren/Pi+Energy.pdf}
  \caption{multiplicity and energy spectrum of the \(\pi^+\)} \label{gr:prim-pip}
\end{figure}
The plots {\eqref{gr:prim-pip}} from the \(\pi^+\) show drastic changes. Through 
the update the probability for no and one particle becomes equal. This leads 
mainly from the processes with Outgoing part 3224,211,-211 and 3212,211 because 
there only one \(\pi^+\) is created and both ratios are increased. The adding 
of the process with the outgoing part 3112,211,211 is the reason for the 
increase of 2-multiplicity events. The padding of the gap near 0.7 GeV comes 
also from a created process, 3122,211,111. The reason for this clear identification 
can be located in the Plot {\eqref{gr:prim-pi}} of the \(\pi\). There the same gap 
is filled. The last peak is obtained from 2112,211. Only two particles were 
created and so the the pion get an very high energy. This process is also 
probable with \(0.3\%\). That's explain the peak. The second peak 
results from 3122,211 because the \(\Lambda\) has  slightly higher mass than 
the n and so the peak is moved backwards. The peak that is very close to the 
\(\Lambda\) peak results from the process 3212,211. The \(\Sigma\) has nearly 
the same mass and so the peaks are very close. The peak with the lowest 
energy belongs to a decay that doesn't changed and so it is not included in 
this thesis. The process is \(Lambda_c^+ \rightarrow \Delta(1232) + \pi^+\) 
with an BR of \(0.3 \%\). The mass of the \(\Delta(1232)\) is higher compared 
to the others and so the peak has to be at a lower position.
\begin{figure}[h]
  \centering
  \includegraphics[width=0.45\textwidth]{../results/identified_states/comp_data/rivet-plots/MeasureLeptonsAndMoreDirectChildren/Pi.pdf}
  \includegraphics[width=0.45\textwidth]{../results/identified_states/comp_data/rivet-plots/MeasureLeptonsAndMoreDirectChildren/PiEnergy.pdf}
  \caption{multiplicity and energy spectrum of the \(\pi\)} \label{gr:prim-pi}
\end{figure}
The picture {\eqref{gr:prim-pi}} is very similar to {\eqref{gr:prim-pip}}. In 
further decays this graph is not visible because the \(\pi\) is not stable.
The peak at the highest energy comes from 2212,111. The peak is nearly on the same 
position because the mass of the proton is nearly the mass of the neutron. 
There exists no decay channel in a \(\Lambda\) baryon for the neutral pion. So 
the pick at this position is missing. The others peak are the same because 
the mass of the charged baryons only slightly different to the correspondending
neutral baryon.
\begin{figure}[h]
  \centering
  \includegraphics[width=0.45\textwidth]{../results/identified_states/comp_data/rivet-plots/MeasureLeptonsAndMoreDirectChildren/K+.pdf}
  \includegraphics[width=0.45\textwidth]{../results/identified_states/comp_data/rivet-plots/MeasureLeptonsAndMoreDirectChildren/K+Energy.pdf}
  \caption{multiplicity and energy spectrum of the \(K^+\)} \label{gr:prim-Kp}
\end{figure}
The diagram {\eqref{gr:prim-Kp}} show only little changes. No processes were 
added and so this resukt is aspected. Through the masses of the other particle 
in the final state the peaks can be connected to the different processes. From 
right to the left peaks the decay partner are \(\Lambda\), \(\Sigma\), \(\Xi\),
\(\Sigma(1385)\) and \(\Xi(1530)\).
\begin{figure}[h]
  \centering
  \includegraphics[width=0.45\textwidth]{../results/identified_states/comp_data/rivet-plots/MeasureLeptonsAndMoreDirectChildren/AntiK.pdf}
  \includegraphics[width=0.45\textwidth]{../results/identified_states/comp_data/rivet-plots/MeasureLeptonsAndMoreDirectChildren/AntiKEnergy.pdf}
  \caption{multiplicity and energy spectrum of the \(\bar{K}\)} \label{gr:prim-Kb}
\end{figure}
\begin{figure}[h]
  \centering
  \includegraphics[width=0.45\textwidth]{../results/identified_states/comp_data/rivet-plots/MeasureLeptonsAndMoreDirectChildren/Ks.pdf}
  \includegraphics[width=0.45\textwidth]{../results/identified_states/comp_data/rivet-plots/MeasureLeptonsAndMoreDirectChildren/KsEnergy.pdf}
  \caption{multiplicity and energy spectrum of the \(K_s\)} \label{gr:prim-Ks}
\end{figure}
The events that are missed in {\eqref{gr:prim-Kb}} are in {\eqref{gr:prim-Ks}}
because the process 2212,-311 was changed to 2212,310 and 
2212,-311,211,-211 to 2212,310,211,-211. This leads to the two 
diagrams. The peak in {\eqref{gr:prim-Ks}} belongs to the proton
\begin{figure}[h]
  \centering
  \includegraphics[width=0.45\textwidth]{../results/identified_states/comp_data/rivet-plots/MeasureLeptonsAndMoreDirectChildren/K.pdf}
  \includegraphics[width=0.45\textwidth]{../results/identified_states/comp_data/rivet-plots/MeasureLeptonsAndMoreDirectChildren/KEnergy.pdf}
  \caption{multiplicity and energy spectrum of the \(K\)} 
\end{figure}
\(\Sigma^+\) is the reason for the high isolated peak and \(\Sigma(1385)+\) for 
the peak on the hill. The diagrams are changed because the branching ratios 
were changed, they decreased
\begin{figure}[h]
  \centering
  \includegraphics[width=0.45\textwidth]{../results/identified_states/comp_data/rivet-plots/MeasureLeptonsAndMoreDirectChildren/AntiPi+.pdf}
  \includegraphics[width=0.45\textwidth]{../results/identified_states/comp_data/rivet-plots/MeasureLeptonsAndMoreDirectChildren/AntiPi+Energy.pdf}
  \caption{multiplicity and energy spectrum of the \(\pi^-\)} \label{gr:prim-pi}
\end{figure}
The decrease of the process 2212,211,211,-211,-211 has an dramatically impact 
of the multiplicity diagram because it is the only one that produces two \(\pi^-\).
The red bulge is a superposition of all little chances because no processes were 
changed so drastically.

\clearpage
\section{Form Factor Models}
All diagrams of the form factors shwo the same behavior. There is no difference 
in the semileptonic decays in electron or muon except of an sharp increase of 
the beginnig of the graph for the muon. But this comes from the higher 
mass of the muon compared to the electron.
\begin{figure}[h]
  \centering
  \includegraphics[width=0.75\textwidth]{../results/formfactor_new/with_old_comp/rivet-plots/LambdaCPlus_semileptonic/LambdaQ2.pdf}
  \caption{Comparison from the already implemented form factor models in leadding order (old)
  and with all factors for \(\Lambda\)} \label{gr:with_old_comp}
\end{figure}

\begin{figure}[h]
  \centering
  \includegraphics[width=0.75\textwidth]{../results/formfactor_new/LCSR_comp/rivet-plots/LambdaCPlus_semileptonic/LambdaQ2.pdf}
  \caption{Comparison from the different parametrization of the LCSR model for \(\Lambda\)} \label{gr:lcsr}
\end{figure}

\begin{figure}[h]
  \centering
  \includegraphics[width=0.75\textwidth]{../results/formfactor_new/QCDSR_comp/rivet-plots/LambdaCPlus_semileptonic/LambdaQ2.pdf}
  \caption{Comparison from the different parametrization of the QCDSR model for \(\Lambda\)} \label{gr:qcdsr}
\end{figure}

\begin{figure}[h]
  \centering
  \includegraphics[width=0.75\textwidth]{../results/formfactor_new/others_comp/rivet-plots/LambdaCPlus_semileptonic/LambdaQ2.pdf}
  \caption{Comparison from some of the self implemented form factor models for \(\Lambda\)} \label{gr:others_comp}
\end{figure}

\begin{figure}[h]
  \centering
  \includegraphics[width=0.75\textwidth]{../results/formfactor_new/others_comp/rivet-plots/LambdaCPlus_semileptonic/NeutronQ2.pdf}
  \caption{Comparison from some of the self implemented form factor models for the neutron} \label{gr:others_comp_n}
\end{figure}

\begin{figure}[h]
  \centering
  \includegraphics[width=0.75\textwidth]{../results/formfactor_new/others_orig_comp/rivet-plots/LambdaCPlus_semileptonic/LambdaQ2.pdf}
  \caption{Comparison from the already implemented form factor models with 
  some of the self implemented ones for \(\Lambda\)} \label{gr:others_orig_comp}
\end{figure}

\begin{figure}[h]
  \centering
  \includegraphics[width=0.75\textwidth]{../results/formfactor_new/others_orig_comp/rivet-plots/LambdaCPlus_semileptonic/NeutronQ2.pdf}
  \caption{Comparison from the already implemented form factor models with 
  some of the self implemented ones for the neutron} \label{gr:others_orig_comp_n}
\end{figure}
