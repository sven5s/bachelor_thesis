\section{Decay table}
The decays of the \(\Lambda_c^+\) were simulated with the old and the new 
decay table. 100000000  decay events were used. This leads to a low 
statistic uncertainty. This section should show the impact of my changes in 
the decay table. Selected inclusive modes are used to show the difference. The 
observables multiplicity and energy are used for the comparison. The multiplicity 
is the number of created particles from the choosen type at the consideres state. The 
energy is measured by the squared four-momentum of the particles from the choosen type.\\
The figures in the following section can also be found in the appendix{\ref{a:primary}}.
These figures only show the measurement of the particles after the primary 
decay of the \(\Lambda_c^+\). The direct impact due to the changes in the decay 
table  is so better visible. The figures for the measurement at the final state 
of the decay chain are in {\ref{a:complete}}.

\begin{figure}[h]
  \centering
  \includegraphics[width=0.45\textwidth]{../results/identified_states_new/comp_data/rivet-plots/MeasureLeptonsAndMoreDirectChildren/AntiElectron.pdf}
  \includegraphics[width=0.45\textwidth]{../results/identified_states_new/comp_data/rivet-plots/MeasureLeptonsAndMoreDirectChildren/AntiElectronEnergy.pdf}
  \caption{multiplicity and energy spectrum of the \(e^+\)} \label{gr:prim-ep}
\end{figure}
\begin{figure}[h]
  \centering
  \includegraphics[width=0.45\textwidth]{../results/identified_states_new/comp_data/rivet-plots/MeasureLeptonsAndMoreDirectChildren/AntiMyon.pdf}
  \includegraphics[width=0.45\textwidth]{../results/identified_states_new/comp_data/rivet-plots/MeasureLeptonsAndMoreDirectChildren/AntiMyonEnergy.pdf}
  \caption{multiplicity and energy spectrum of the \(\mu^+\)} \label{gr:prim-mup}
\end{figure}
In the figures {\ref{gr:prim-ep}} and {\ref{gr:prim-mup}} you can see that 
the change of the branching ratios of the semileptonic decays 
resulting in a nearly equal multiplicity distribution but a significant different 
energy spectrum for both leptons. The cut-off at the beginning of the muon 
energy spectrum comes from the bigger rest mass than the electron.This energy is 
needed to create a detectable on-shell muon.\\
The two main processes to create leptons are the semileptonic decays
\begin{equation}
  \Lambda_c^+ \rightarrow \Lambda + l^+ + \nu_l \nonumber
\end{equation} and 
\begin{equation}
  \Lambda_c^+ \rightarrow n+ l^+ + \nu_l. \nonumber
\end{equation}
Only the branching ratios for the processes with the \(\Lambda\) in the final state were changed. 
The higher rest mass of the \(\Lambda\) compared to the neutron leads to leptons 
with a lower energy. The braching ratios for the neutron decays are the same as 
before but the branching ratio for the semileptonic \(\Lambda\) decays were increased. T
he increase of the branching ratio is visible as the energy shift in the energy spectrum.\\
The multiplicity didn't changed much because both processes have the same 
multiplicity. The probability for one positron becomes a little lower. This 
leads from the lower increase of the branching ratio of the electron compared 
to the muon.\\
These changes in the primary decay leads also to changes in the further decays.
The figures for the measurement of the particles after the whole decay chain 
in {\ref{a:complete}} look really the same.
\begin{figure}[h]
  \centering
  \includegraphics[width=0.45\textwidth]{../results/identified_states_new/comp_data/rivet-plots/MeasureLeptonsAndMoreDirectChildren/Pi+.pdf}
  \includegraphics[width=0.45\textwidth]{../results/identified_states_new/comp_data/rivet-plots/MeasureLeptonsAndMoreDirectChildren/Pi+Energy.pdf}
  \caption{multiplicity and energy spectrum of the \(\pi^+\)} \label{gr:prim-pip}
\end{figure}
The plots {\ref{gr:prim-pip}} from the \(\pi^+\) show drastic changes. This 
changes will be discussed in the following paragraph. Due to the update of 
the decay table the probability for multiplicity zero and one processes
becomes equal. This leads mainly from the processes
\begin{equation}
  \Lambda_c^+ \rightarrow \Sigma(1385)^+ + \pi^+ + \pi^- \nonumber
\end{equation} and 
\begin{equation}
  \Lambda_c^+ \rightarrow \Sigma + \pi^+. \nonumber
\end{equation}
In these processes one \(\pi^+\) is created and both branching ratios has been increased. This explains the 
increase of the probability for the production of on pion.\\
The decay channel
\begin{equation}
  \Lambda_c^+ \rightarrow \Sigma^- + \pi^+ + \pi^+ \nonumber
\end{equation}
was added to the decay table. This process leads to the increase of the probability for 
pion multiplicity two.
\par
The new decay channel
\begin{equation}
  \Lambda_c^+ \rightarrow \Lambda + \pi^+ + \pi \nonumber
\end{equation}
is the reason why the gap near 0.7 GeV is filled. It is the only reason for this 
change because in {\ref{gr:prim-pi}} happens the same thing for the \(\pi\).
And the only new process with a \(\pi^+\) and a \(\pi\) in the final state 
is the process
\begin{equation}
  \Lambda_c^+ \rightarrow \Lambda + \pi^+ + \pi \nonumber
\end{equation}
The peaks in the energy spectrum can be identified with specific decay channels.
This channels are \( 1 \rightarrow 2 \) processes because in such a process the 
dynamics of the end products is well determined due to momentum conversation.
The peak with the highest energy is obtained from 
\begin{equation}
  \Lambda_c^+ \rightarrow n + \pi^+ \nonumber
\end{equation}
because the proton have the lowest mass from all products of \( 1 \rightarrow 2 \) 
processes with a \(\pi^+\) in the final state. The second peak comes from 
\begin{equation}
  \Lambda_c^+ \rightarrow \Lambda + \pi^+, \nonumber
\end{equation}
the third from 
\begin{equation}
  \Lambda_c^+ \rightarrow \Sigma + \pi^+ \nonumber
\end{equation}
and the fourth from 
\begin{equation}
  \Lambda_c^+ \rightarrow \Delta(1232) + \pi^+. \nonumber
\end{equation}
The peak from the \(\Delta(1232)\) is at the lowest postion because the mass of 
this particle is the highest compared to the others.

\begin{figure}[h]
  \centering
  \includegraphics[width=0.45\textwidth]{../results/identified_states_new/comp_data/rivet-plots/MeasureLeptonsAndMoreDirectChildren/Pi.pdf}
  \includegraphics[width=0.45\textwidth]{../results/identified_states_new/comp_data/rivet-plots/MeasureLeptonsAndMoreDirectChildren/PiEnergy.pdf}
  \caption{multiplicity and energy spectrum of the \(\pi\)} \label{gr:prim-pi}
\end{figure}
The diagram {\ref{gr:prim-pi}} is very similar to {\ref{gr:prim-pip}}. For a 
complete decay chain {\ref{a:complete}} the distribution for the \(\pi\) 
disappears because it is unstable.\\
In the further consideration only the other end product for \( 1 \rightarrow 2 \) process
that leads to a peak is used to characterize the decay. This is simpler and the 
other end product will always be the regarded particle of the energy spectrum. 
The other end product besides the \(\pi\) has to be a positive charged particle. 
The family of baryons for the decay will be the same only a charged particle is 
picked up. For the \(\Lambda\) baryons exist only the \(\Lambda_t^+\) besides 
the \(\Lambda_c^+\) with a positive charge. Both are no decay candidates because 
their masses are to high. So at this position there will be no peak. The 
peaks from the right to the left results from the following decays:
n, \(\Sigma^+\), \(\Delta(1232)^+\).
\begin{figure}[h]
  \centering
  \includegraphics[width=0.45\textwidth]{../results/identified_states_new/comp_data/rivet-plots/MeasureLeptonsAndMoreDirectChildren/K+.pdf}
  \includegraphics[width=0.45\textwidth]{../results/identified_states_new/comp_data/rivet-plots/MeasureLeptonsAndMoreDirectChildren/K+Energy.pdf}
  \caption{multiplicity and energy spectrum of the \(K^+\)} \label{gr:prim-Kp}
\end{figure}
The diagram {\ref{gr:prim-Kp}} show only little changes. No decay channels 
with a \(K^+\) in the final state were added and so this result is aspected. 
Due to the masses of the other particle in the final state the peaks can be 
connected to the different decay channels. From right to the left the peaks 
results from \( 1 \rightarrow 2 \) decays with a \(\Lambda\), \(\Sigma\), \(\Xi\),
\(\Sigma(1385)\) or a \(\Xi(1530)\) and a \(K^+\) in the final state.
\begin{figure}[h]
  \centering
  \includegraphics[width=0.45\textwidth]{../results/identified_states_new/comp_data/rivet-plots/MeasureLeptonsAndMoreDirectChildren/AntiK.pdf}
  \includegraphics[width=0.45\textwidth]{../results/identified_states_new/comp_data/rivet-plots/MeasureLeptonsAndMoreDirectChildren/AntiKEnergy.pdf}
  \caption{multiplicity and energy spectrum of the \(\bar{K}\)} \label{gr:prim-Kb}
\end{figure}
\begin{figure}[h]
  \centering
  \includegraphics[width=0.45\textwidth]{../results/identified_states_new/comp_data/rivet-plots/MeasureLeptonsAndMoreDirectChildren/Ks.pdf}
  \includegraphics[width=0.45\textwidth]{../results/identified_states_new/comp_data/rivet-plots/MeasureLeptonsAndMoreDirectChildren/KsEnergy.pdf}
  \caption{multiplicity and energy spectrum of the \(K_s\)} \label{gr:prim-Ks}
\end{figure}
The events that are missed in {\ref{gr:prim-Kb}} are in {\ref{gr:prim-Ks}}
because the process
\begin{equation}
  \Lambda_c^+ \rightarrow P^+ + K_b \nonumber
\end{equation}
was changed to 
\begin{equation}
  \Lambda_c^+ \rightarrow P^+ + K_s \nonumber
\end{equation}.
The process
\begin{equation}
  \Lambda_c^+ \rightarrow P^+ + K_s + \pi^+ + \pi^- \nonumber
\end{equation}
was added as a decay channel. This leads to the distribution in the two diagrams. 
The peak in {\ref{gr:prim-Ks}} belongs to the proton.
\begin{figure}[h]
  \centering
  \includegraphics[width=0.45\textwidth]{../results/identified_states_new/comp_data/rivet-plots/MeasureLeptonsAndMoreDirectChildren/K.pdf}
  \includegraphics[width=0.45\textwidth]{../results/identified_states_new/comp_data/rivet-plots/MeasureLeptonsAndMoreDirectChildren/KEnergy.pdf}
  \caption{multiplicity and energy spectrum of the \(K\)} 
\end{figure}
The \( 1 \rightarrow 2 \) decays with a \(\Sigma^+\) is the reason for the high 
isolated peak and \(\Sigma(1385)+\) for the peak on the hill. The diagrams are 
changed slightly  because the branching ratios were changed. They were decreased.
\begin{figure}[h]
  \centering
  \includegraphics[width=0.45\textwidth]{../results/identified_states_new/comp_data/rivet-plots/MeasureLeptonsAndMoreDirectChildren/AntiPi+.pdf}
  \includegraphics[width=0.45\textwidth]{../results/identified_states_new/comp_data/rivet-plots/MeasureLeptonsAndMoreDirectChildren/AntiPi+Energy.pdf}
  \caption{multiplicity and energy spectrum of the \(\pi^-\)} \label{gr:prim-pi}
\end{figure}
The decrease of the process
\begin{equation}
  \Lambda_c^+ \rightarrow P^+ + \pi^+ + \pi^+ + \pi^- + \pi^- \nonumber
\end{equation}
has an dramatically impact of the multiplicity diagram because it is the only process 
that produces two \(\pi^-\). The red bulge is a superposition of all little changes 
because no processes were changed so drastically.


\clearpage
\section{Form Factor Models}
All diagrams of the form factors show the same behavior, there is no difference 
in the semileptonic decays in electron or muon except of the cut-off due to 
the higher muon mass compared to the elecron mass. So the figures for 
the electrons were used in the following sections.\\
The goal of this section is to find the best form factor model that fits 
the experimental results {\ref{gr:data}} from the CLEO II detector{\cite{data}}.
\begin{figure}[h]
  \centering
  \includegraphics[width=0.6\textwidth]{../results/formfactor_new/exp_data.pdf}
  \caption{Experimental results from the CLEO II detector for the 
  \(\Lambda_c^+ \rightarrow \Lambda + e^+ + \nu_e\) decay} \label{gr:data}
\end{figure}
The CLEO II detector has measured the \(\Lambda_c^+ \rightarrow \Lambda + e^+ + \nu_e\)
 decay channel. The observables for the comparison have to be defined first.

\subsection{Observables}
Two observables are adapted from simulations of the \texttt{BELLE} experiment 
to compare the different form factors.
The first is \(q^2\) which is defined as
\begin{equation}
  q^2 = \left( p_{\Lambda_c} - p_B \right)^2a \label{eq:q2}
\end{equation}
and the second is the recoil of the W boson:
\begin{equation}
  w = \frac{p^2_{\Lambda_c} +  p^2_B - \left( p_{\Lambda_c} - p_B \right)^2}
  {2 \cdot \sqrt{p^2_{\Lambda_c}} \cdot \sqrt{p^2_B}}. \label{eq:w}
\end{equation}
The experimental data use the first observable {\eqref{eq:q2}}. So only this 
one will be compared. The figures for the recoil of the W boson can be found 
in the appendix{\ref{a:ffm}}

\subsection{Previous form factor models}
\begin{figure}[h]
  \centering
  \includegraphics[width=0.6\textwidth]{../results/formfactor_new/with_old_comp/rivet-plots/LambdaCPlus_semileptonic/LambdaElectronQ2.pdf}
  \caption{Comparison from the already implemented form factor models in leadding order (old)
  and with all factors for \(\Lambda\)} \label{gr:with_old_comp}
\end{figure}

For the \(\Lambda_c^+\) alrady exists four form factor models. The harmonic 
oszillator non relativistic (HONR), the harmonic oszillator semi relativistic 
(HOSR), the sturmian non relativistic (STNR) and the the sturmian semi 
relativistic (STSR) model. Harmonic oszillator and sturmian are to different 
expansion bases for the wave function. {\cite{prev}} \\
But for this models only the first form factor were used to calculate the 
transition matrix. These are the lines that marked with old. The other four 
lines are made with the whole number of form factors. You can see that the shape 
of the curves with only one form factor are more flat compared 
to the curves with all form factors. So all form factors have a not neglectable 
influence to the transistion matrix element and the leading order term is not 
enough for the simulation.\\ 
For the following form factor models all form factors are used to get accurate 
results.

\clearpage
\subsection{\(\Lambda_c^+ \rightarrow \Lambda + l^+ + \nu_l\)}
\begin{figure}[h]
  \centering
  \includegraphics[width=0.6\textwidth]{../results/formfactor_new/LCSR_comp/rivet-plots/LambdaCPlus_semileptonic/LambdaElectronQ2.pdf}
  \caption{Comparison from the different parametrization of the LCSR model for \(\Lambda\)} \label{gr:lcsr}
\end{figure}
The both parametrizations of the light-cone sum rule in {\ref{gr:lcsr}} for 
different twist show similar properties. You can see that this model did not 
fit the experimental results.\\


\begin{figure}[h]
  \centering
  \includegraphics[width=0.6\textwidth]{../results/formfactor_new/QCDSR_comp/rivet-plots/LambdaCPlus_semileptonic/LambdaElectronQ2.pdf}
  \caption{Comparison from the different parametrization of the QCDSR model for \(\Lambda\)} \label{gr:qcdsr}
\end{figure}
Triangular and Rectangular are different continuums models in figure 
{\ref{gr:qcdsr}} used by the QCD sum rule  model and the number is the value 
of \(\kappa\). This is an parameter of the parametrization.
The behavior of the QCD sum rule model  in {\ref{gr:qcdsr}} is very different 
for the two classes of parametrizations, Pole and Full. The pole 
parametrizations themselves are very similar. They only have little deviations 
to each other. The general behavior of the pole parametrizations is flat and 
has no thin maximum.\\ 
The full parametrizations on the other hand have a clear maxmimum and a more steep 
slope. The full parametrization with the continuums model triangular and \(\kappa = 2\) is an 
exception. The full parametrizations in the rectangular continuums model 
are very close and a dstinction is not possible.\\

\begin{figure}[h]
  \centering
  \includegraphics[width=0.6\textwidth]{../results/formfactor_new/others_comp/rivet-plots/LambdaCPlus_semileptonic/LambdaElectronQ2.pdf}
  \caption{Comparison from some of the self implemented form factor models for \(\Lambda\)} \label{gr:others_comp}
\end{figure}
The diagram {\ref{gr:others_comp}} shows the big differences between the 
different models. The spread between the curves is very high. The CCQM and 
the RQM are very similar. The QCDSR model with triangular continuums model 
and \(\kappa = 2\) fits the experimental data best.\\

\begin{figure}[h]
  \centering
  \includegraphics[width=0.6\textwidth]{../results/formfactor_new/others_orig_comp/rivet-plots/LambdaCPlus_semileptonic/LambdaElectronQ2.pdf}
  \caption{Comparison from the already implemented form factor models with 
  some of the self implemented ones for \(\Lambda\)} \label{gr:others_orig_comp}
\end{figure}
In {\eqref{gr:others_orig_comp}} is a comparison with the original implemented 
form factor models. The LCS is similar to the HONR.\\
The STSR fits the measurement also well. But the first bin is not matching 
very well and also the maximum is to low.\\

So the model that fits the decay \(\Lambda_c^+ \rightarrow \Lambda + l^+ + \nu_l\) 
the best is the QCDSR model with triangular continuums model and \(\kappa = 2\).


\clearpage
\subsection{\(\Lambda_c^+ \rightarrow n + l^+ + \nu_l\)}
\begin{figure}[h]
  \centering
  \includegraphics[width=0.6\textwidth]{../results/formfactor_new/others_orig_comp/rivet-plots/LambdaCPlus_semileptonic/NeutronElectronQ2.pdf}
  \caption{Comparison from the already implemented form factor models with 
  the self implemented ones for the neutron} \label{gr:others_orig_comp_n}
\end{figure}
The form factor models {\ref{gr:others_orig_comp_n}} shows in general a more 
flat shape compared to the results for the \(\Lambda_c^+ \rightarrow \Lambda + l^+ + \nu_l\) 
decay. Only the CCQM and the RQM have an computation for the neutron.
The already implemented HOSR was also used. No experimental data exists for this 
decay.\\ 
The models for the neutron didn't have such big deviations like the ones for the \(\Lambda\). 
So the choice of the model is not so important for the neutron because all of this 
will give similar results due to the similar shape.

