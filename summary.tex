This bachelor thesis gave a little contribution to world biggest epxeriment 
through the update of the branching ratios of the \(\Lambda_c^+\). Every 
Baryon is important because with everyone the modeling of the processess 
in the nature becomes more accurate. Through more exact modeling the 
comparison between our laws of nature, in this special case the 
standard model of particle physics, and the reallity becomes better and 
better again. So discrepancies are discoverable and the theory can be 
improved.\\
This work shows that the position of th peaks in the energy spectrum are 
correlated to the masses of the other particles in the final state. These peaks 
occurs only with two particles in the final state. This comes from the 
determination of the four-momentum in this case. The particle have no other 
choice to put the momentum elsewhere.\\
There exists a lot of different form factor models and the behavior of them 
can partially very different, flat or steep run, only a comparison with 
experimental data can clarify. This leads to a lot of measurements because 
every particle have a slightly different behavior when its decays. But 
only with these amount of data a clear answer for every particle is possible.
\par
\par
The non relativistic quak model \cite{NRQM} was introduced am implement with 
this thesis in \texttt{Sherpa}. For the exicted \(\Lambda\) states didn't 
exist so much experimental data. The concentration has layed on the 
other models because it would be easier to get data. But as a continuation to 
this work this model can be tested. Further an implementation of the form 
factor model from Lattice QCD \cite{Lattice_QCD} would be a great work. 
The Lattice QCD is a modern method of the QCD and used in many aspect.
