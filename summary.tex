In this thesis I have examined the semileptonic decays of the \(\Lambda_c^+\).
First I have updated the decay table with the latest branching ratios and 
decay channels from the PDG. We have seen that the changes in the decay table have an 
great impact of the multiplicity and the energy spectrum of the particles in 
the final state.\\
Further I have searched for different form factor models and found five 
different models. These are the covariant confined quark model, the non 
relativistic quark model, a form factor model from the light-cone sum rule, 
the relativistic quark model and a model from the QCD sum rule. I have made 
simulations with four of the models. I did not used the non relativistic quak model.
These simulations have shown that the QCDSR model with triangular continuums model 
and \(\kappa = 2\) fits best the experimental data for the 
\(\Lambda_c^+ \rightarrow \Lambda + l^+ + \nu_l\) decay. The destinction between 
the different models are very high. It was also shown that 
the differences between the tested form factors models for the 
\(\Lambda_c^+ \rightarrow n + l^+ + \nu_l\) decay are very small.\\

The non relativistic quak model \cite{NRQM} was introduced and implemented with 
this thesis in \texttt{Sherpa} but not tested because for the exicted \(\Lambda\) states 
 not so much experimental data exists. This model could be tested and verified
 in the future.
An implementation of the form factor model from Lattice QCD \cite{Lattice_QCD} 
would also be a future work. 
The Lattice QCD is a modern method of the QCD and used in many aspect. So 
the result from this form factor model would be very interesting.
